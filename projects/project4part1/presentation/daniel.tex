\section{\cry{} Description}

\begin{frame}
\frametitle{Project Description}
  \begin{itemize}
  \item The goal of our project is to develop a framework for
    cryptographers to easily test their \cs s in a
    secure environment.
  \item Users will be able to implement their
    \cs{} securely in Cry, and upon testing it, they will
    receive a report that will explain results of a particular
    test as well as ways to improve the system.
  \end{itemize}
\end{frame}

\begin{frame}
\frametitle{Description cont.}
  \begin{itemize}
  \item The will give details on various aspects of the tested system.
    These details will include time (or estimated time) required
    to encrypt, decrypt, and cryptanalyze.
  \item When analyzing a
    cryptanalysis method, Cry will hope to give some useful
    suggestions as to why the system was broken in the given
    time, and what could be used to improve it in further
    iterations.
  \end{itemize}
\end{frame}

%\begin{frame}
%\frametitle{my frame title}
%If you want to show code from another file:
%\codes{\lstinputlisting[language=c++]
  %{../code/daniel/daniel.cc}}
  % path relative to `project4part1/presentation/`
%\end{frame}

%%%%%%%%%%%%%%%%%%%%%%%%%%%%%%%%%%%%%%%%%%%%%%%%%%%%%%%%%%%%

\section{GMP Installation}

\begin{frame}
\frametitle{Installation}
\begin{itemize}
\item \url{https://gmplib.org/\#DOWNLOAD}
\item Initially, a download is obviously needed. For Linux
  systems, one can simply go to the URL given above, and
  download the .tar file. This does not work for Mac.
  Instead, following the instructions we have posted on our
  Github, GMP installation can be done using the \code{brew
  install gmp} command.
\item On Linux systems, after downloading the file, a
  simple \code{./configure} and then \code{make install}
  will install GMP.
\end{itemize}
\end{frame}

%%%%%%%%%%%%%%%%%%%%%%%%%%%%%%%%%%%%%%%%%%%%%%%%%%%%%%%%%%%%

\section{GMP Basics}

\begin{frame}
\frametitle{GMP Basics}
\begin{itemize}
\item \url{https://gmplib.org/manual/GMP-Basics.html\#GMP-Basics}
\item In general, it is important to use only specific
  macros and data types that are documented by GMP. This is
  important for compatibility with other versions of the
  library. The URL listed gives some insight into helpful
  guidelines to follow for aspects of a project that uses
  GMP such as headers, libraries, parameter conventions,
  debugging, useful macros, as well as others. If there is
  any confusing regarding these topics it is best to refer
  to the GMP basics section in the manual.
\end{itemize}
\end{frame}
