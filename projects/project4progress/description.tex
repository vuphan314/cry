\section{Project Description}
(\md)

% 1. Short project description -- remind me.

%%%%%%%%%%%%%%%%%%%%%%%%%%%%%%%%%%%%%%%%%%%%%%%%%%%%%%%%%%%%

\subsection{Overview}
\cry{} is a framework whose aim is to provide developers of
new cryptosystems a convenient way to benchmark the
approximate effectiveness of their cryptosystem in
development.

\subsection{External Systems}
\cry{} will make use of the GitHub version control system
as the means of message transportation. Also, GitHub will
provide inherent tamper protection, as any changes made
to the repository will be evident. Implied is that each
actor (sender, receiver, and eavesdropper) will have access
to this said repository.

\subsection{Use}
The developers will be required to modify \cry's source
code in several ways. For one, they will need to implement
their cryptosystem as a subclass of our \code{Cryptosystem}
class. As part of this process, they must override the
\code{Cryptosystem} class' functions generateKeys, encrypt,
decrypt, and cryptanalyze.

\subsection{Other Uses}
\cry{} will allow users to send, receive, and perform
cryptanalysis on the messages in an effort to view the
plaintext without having access to the private key of
the receiver.
