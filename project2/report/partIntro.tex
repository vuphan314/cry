\section{Introduction}
\md

%%%%%%%%%%%%%%%%%%%%%%%%%%%%%%%%%%%%%%%%%%%%%%%%%%%%%%%%%%%%

\subsection{Purpose}

This document outlines the requirements that \tc{} will meet
in the development of \cry{}. The target audience is \cg s,
as well as crypto-nerds who want to use
\cry{} for exchanging messages.

%%%%%%%%%%%%%%%%%%%%%%%%%%%%%%%%%%%%%%%%%%%%%%%%%%%%%%%%%%%%

\subsection{Scope}

The primary purpose of \cry{} is to give \cg s the ability
to benchmark the \cs s they are developing. The secondary
purpose of \cry{} is to allow users to encrypt and decrypt
data.

%%%%%%%%%%%%%%%%%%%%%%%%%%%%%%%%%%%%%%%%%%%%%%%%%%%%%%%%%%%%

\subsection{Definitions, acronyms, and abbreviations}

\begin{itemize}
\item \cry{}: The cryptographic benchmarking system under
      development.
\item \tc{}: The team responsible for developing \cry{}.
\item \cg : The intended audience for \cry{}.
\end{itemize}

%%%%%%%%%%%%%%%%%%%%%%%%%%%%%%%%%%%%%%%%%%%%%%%%%%%%%%%%%%%%

\subsection{References}

\begin{itemize}
  \item \textbf{GNU Multiple Precision Arithmetic Library:}
        https://gmplib.org/
  \item \textbf{Msieve:} https://github.com/radii/msieve
\end{itemize}

%%%%%%%%%%%%%%%%%%%%%%%%%%%%%%%%%%%%%%%%%%%%%%%%%%%%%%%%%%%%

\subsection{Overview}

An overall description of the system can be found in section
2. Descriptions of different interfaces and c++ function
prototypes can be found in section 3.

%%%%%%%%%%%%%%%%%%%%%%%%%%%%%%%%%%%%%%%%%%%%%%%%%%%%%%%%%%%%
