\section{Introduction}
(by \md)

%%%%%%%%%%%%%%%%%%%%%%%%%%%%%%%%%%%%%%%%%%%%%%%%%%%%%%%%%%%%

\subsection{Purpose}

This document outlines the requirements that \tc{} will meet
in the development of \cry{}. The target audience is \cg s,
as well as crypto-nerds who want to use
\cry{} for exchanging messages.

%%%%%%%%%%%%%%%%%%%%%%%%%%%%%%%%%%%%%%%%%%%%%%%%%%%%%%%%%%%%

\subsection{Scope}

The primary purpose of \cry{} is to give \cg s the ability
to benchmark the \cs s they are developing. The secondary
purpose of \cry{} is to allow users to encrypt and decrypt
data.

%%%%%%%%%%%%%%%%%%%%%%%%%%%%%%%%%%%%%%%%%%%%%%%%%%%%%%%%%%%%

\subsection{Definitions, Acronyms, and Abbreviations}

\begin{itemize}
\item \cry{}: The \cf{}  under development.
\item \tc{}: The team responsible for developing \cry{}.
\item Cryptographer : The intended audience for \cry{}.
\end{itemize}

%%%%%%%%%%%%%%%%%%%%%%%%%%%%%%%%%%%%%%%%%%%%%%%%%%%%%%%%%%%%

\subsection{References}

\begin{itemize}
  \item \textbf{GNU Multiple Precision Arithmetic Library:}
        \url{https://gmplib.org/}
  \item \textbf{Msieve:} \url{https://github.com/radii/msieve}
\end{itemize}

%%%%%%%%%%%%%%%%%%%%%%%%%%%%%%%%%%%%%%%%%%%%%%%%%%%%%%%%%%%%

\subsection{Overview}

An overall description of the system can be found in section
2. Descriptions of different interfaces and \code{C++} function
prototypes can be found in section 3.

%%%%%%%%%%%%%%%%%%%%%%%%%%%%%%%%%%%%%%%%%%%%%%%%%%%%%%%%%%%%
