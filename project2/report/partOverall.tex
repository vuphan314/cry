\section{Overall description}
\dd

%%%%%%%%%%%%%%%%%%%%%%%%%%%%%%%%%%%%%%%%%%%%%%%%%%%%%%%%%%%%

\subsection{Product perspective}

\cry{} will be implemented as a stand-alone framework, with built in libraries updated as needed. No other frameworks
or applications are needed to run cry: it is a standalone application. It will be used as a tool for cryptographers and
developers alike, and all systems will be self-contained in the framework.
  \subsection{User Interface}
  
  \cry{} will be used as a command line application, being accessed with the command 'cry'.

%%%%%%%%%%%%%%%%%%%%%%%%%%%%%%%%%%%%%%%%%%%%%%%%%%%%%%%%%%%%

\subsection{Product functions}

The essential functions of \cry{} can be broken into two separate parts: testing and reporting.
  \subsection{Testing}
  \begin{itemize}
    \item Users will be able to develope new cryptosystems
    \item Users will be able to develope new eavesdropping methods.
    \item Users will be able to develope new methods for breaking a cryptosystem.
    \item Using a combination of cryptosystems, eavesdropping, and breaking, users will have the ability
    to test systems' security.
  \end{itemize}
  \subsection{Reporting}
  \begin{itemize}
    \item Upon performing a test, users will receive feedback on their methods.
    \item Reports will be shown to suggest the security of the cryptosystem and to give helpful feedback
    in the area of weaknesses in the cryptosystem.
  \end{itemize}

%%%%%%%%%%%%%%%%%%%%%%%%%%%%%%%%%%%%%%%%%%%%%%%%%%%%%%%%%%%%

\subsection{User characteristics}

User of \cry{} will most likely have a medium to high level of experience with cryptosystems. This is not a requirement,
it is an open application. However, \cry{} aims to aid the developing cryptosystems, and this is implies a high level of experience.
If more complicated libraries are implemented, users will need a high level of experience to uderstand these methods and therefore apply
them.

%%%%%%%%%%%%%%%%%%%%%%%%%%%%%%%%%%%%%%%%%%%%%%%%%%%%%%%%%%%%

\subsection{Constraints}

\cry{} will need certain base requirements
\begin{itemize}
  \item Basic memory and CPU availability
  \item Command line permissions
\end{itemize}

In addition to baseline requirements, parallel operation and interfacing with other applications may become necessary if
future library additions dictate such.

%%%%%%%%%%%%%%%%%%%%%%%%%%%%%%%%%%%%%%%%%%%%%%%%%%%%%%%%%%%%

\subsection{Assumptions and dependencies}

Few assumptions are needed as \cry{} will run on all operating systems and is a standalone framework. The single assumption to be
made, as mentioned in constraints, is that certain permissions may be needed from the command line. These are assumed to be available for all users.

%%%%%%%%%%%%%%%%%%%%%%%%%%%%%%%%%%%%%%%%%%%%%%%%%%%%%%%%%%%%
