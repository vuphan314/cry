% a comment starts with %
% preferred max line length: 60-character
% the line below is exactly 60-character
%%%%%%%%%%%%%%%%%%%%%%%%%%%%%%%%%%%%%%%%%%%%%%%%%%%%%%%%%%%%

\section{Introduction}

%%%%%%%%%%%%%%%%%%%%%%%%%%%%%%%%%%%%%%%%%%%%%%%%%%%%%%%%%%%%

\subsection{Scope}

\begin{frame}
\frametitle{Scope}
\begin{itemize}
\item \cry{} will allow \cg s to quickly develop new \cs{}s
\begin{itemize}
  \item It will do so by making testing and benchmarking
  easier
\end{itemize}
\item \cry{} will also allow the encryption/decryption of
data
\end{itemize}
\end{frame}

%%%%%%%%%%%%%%%%%%%%%%%%%%%%%%%%%%%%%%%%%%%%%%%%%%%%%%%%%%%%

\subsection{Definitions, Acronyms, and Abbreviations}

\begin{frame}
\frametitle{Definitions}
\begin{itemize}
\item \cry{}: the \cf{} under development
\item \tc{}: the team responsible for the
  development of \cry{}
\item Cryptographers: the target audience of \cry
\end{itemize}
\end{frame}

%%%%%%%%%%%%%%%%%%%%%%%%%%%%%%%%%%%%%%%%%%%%%%%%%%%%%%%%%%%%

\subsection{References}

\begin{frame}
\frametitle{References}
\begin{itemize}
\item \textbf{GMP (GNU Multiple Precision arithmetic library)}:
  \url{https://gmplib.org/}
\item \textbf{Msieve (General Number Field Sieve
  integer factorization library)}:
  \url{https://github.com/radii/msieve}
\end{itemize}
\end{frame}

%%%%%%%%%%%%%%%%%%%%%%%%%%%%%%%%%%%%%%%%%%%%%%%%%%%%%%%%%%%%
