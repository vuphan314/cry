\section{Specific Requirements}

%%%%%%%%%%%%%%%%%%%%%%%%%%%%%%%%%%%%%%%%%%%%%%%%%%%%%%%%%%%%

\subsection{Interface}

\begin{frame}
\begin{itemize}
\item \alice{} wants to send a confidential message to \bob.
\item \eve{} wants to eavesdrop that message.
\end{itemize}
\end{frame}

%%%%%%%%%%%%%%%%%%%%%%%%%%%%%%%%%%%%%%%%%%%%%%%%%%%%%%%%%%%%

\subsection{Performance}

\begin{frame}
\frametitle{Minimum hardware}
\begin{tabular}{l|l}
RAM & 4 GB \\ \hline
CPU & 1.5 GHz
\end{tabular}
\end{frame}

\subsubsection{Key Generation}

\begin{frame}
\frametitle{\bob}
Input: \generatekeysin \medskip
Output: \generatekeysout
\end{frame}

\subsubsection{Encryption}

\begin{frame}
\frametitle{\alice}
Input: \encryptin \medskip
Output: \encryptout \medskip
Requirements:
\begin{itemize}
\item \plaintextarg{} is an obviously meaningful string,
  such as \plaintextex
\item \ciphertextarg{} is an apparently meaningless string,
  such as \ciphertextex
\end{itemize}
\end{frame}

\subsubsection{Decryption}
\begin{frame}
\frametitle{\bob}
Input: \decryptin \medskip
Output: \decryptout
\end{frame}

\subsubsection{Cryptanalysis}
\begin{frame}
\frametitle{\eve}
Input: \cryptanalyzein \medskip
Output: \cryptanalyzeout
\end{frame}

%%%%%%%%%%%%%%%%%%%%%%%%%%%%%%%%%%%%%%%%%%%%%%%%%%%%%%%%%%%%

\subsection{Classes}
\lstset{language=C++} % basicstyle=\ttfamily
\newcommand{\csh}{../../src/\cs/\cs.h}

\begin{frame}
\frametitle{\code{\cs.h}}
\codes{\lstinputlisting[firstline=10, lastline=12]{\csh}}
\end{frame}

\begin{frame}
\codes{\lstinputlisting[firstline=16, lastline=29]{\csh}}
\end{frame}

% %%%%%%%%%%%%%%%%%%%%%%%%%%%%%%%%%%%%%%%%%%%%%%%%%%%%%%%%%%%%
