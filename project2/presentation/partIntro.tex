% a comment starts with %
% preferred max line length: 60-character
% the line below is exactly 60-character
%%%%%%%%%%%%%%%%%%%%%%%%%%%%%%%%%%%%%%%%%%%%%%%%%%%%%%%%%%%%

\section{Introduction}

%%%%%%%%%%%%%%%%%%%%%%%%%%%%%%%%%%%%%%%%%%%%%%%%%%%%%%%%%%%%

\subsection{Scope}

\begin{frame}
\frametitle{Scope}
\begin{itemize}
\item \cry{} will allow \cg s to quickly develop new \cs{} s
\begin{itemize}
  \item It will do so by making testing and benchmarking
  easier
\end{itemize}
\item \cry{} will also allow the encryption/decryption of
data
\end{itemize}
\end{frame}

%%%%%%%%%%%%%%%%%%%%%%%%%%%%%%%%%%%%%%%%%%%%%%%%%%%%%%%%%%%%

\subsection{Definitions, acronyms, and abbreviations}

\begin{frame}
\frametitle{Definitions}
\begin{itemize}
\item \textbf{\cry{}:} The cryptographic benchmarking system
under development
\item \textbf{\tc{}:} The team responsible for the
development of \cry{}
\item \textbf{\cg:} The target audience for \cry{}
\end{itemize}
\end{frame}

%%%%%%%%%%%%%%%%%%%%%%%%%%%%%%%%%%%%%%%%%%%%%%%%%%%%%%%%%%%%

\subsection{References}

\begin{frame}
\frametitle{References}
\begin{itemize}
  \item \textbf{GNU Multiple Precision Arithmetic Library:}
        https://gmplib.org/
  \item \textbf{Msieve:} https://github.com/radii/msieve
\end{itemize}
\end{frame}

%%%%%%%%%%%%%%%%%%%%%%%%%%%%%%%%%%%%%%%%%%%%%%%%%%%%%%%%%%%%
