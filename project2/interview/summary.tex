(by \dd) \bigskip

All participants were ok with being interviewed as well as
having audio recorded. All participants
agreed that it would be helpful to have more detail on the
report, as oppose to a simple measurement of time. The
additional report features included a simple number scale
(1-10), a specific amount of time specified by the user,
possible weaknesses of the algorithm (i.e. long binary
strings of 1's or 0's or possibly whether the algorithm was
particularly susceptible to attacks other than brute force).
In addition, one participant stated that they would like
information regarding the RAM and CPU power that it took (or
would take) to break the given cryptosystem. All
participants unanimously agreed that other algorithms would
be helpful to have as a baseline or comparison, with AES
being specifically named by all. Additional algorithms were
legacy encryption algorithms such as RC's and DES as well as
legacy hashing algorithms such as MD's and previous SHA's.
The additional features of \cry{} were different among the
interviewees and added more features to consider on
implementation. One popular opinion was additional features
of the report. Certain baselines (calculations per minute,
etc.) as well as helpful graphs were mentioned. Another
additional feature proposed was to have a good random number
generator within the program which uses high entropy sources
for seeds (not system time like traditional RNGs). Also
possible feedback on potential improvements was voiced. For
instance, upon completion of a test, a user would be
notified whether a small, medium, or large amount of work
and changes to their system would be able to solve certain
vulnerabilities found. These would hopefully provide possible
solutions or, in extreme cases, suggest a total rewrite of
the system. Finally, suggestions for changes were made by
participants. Most were small and in addition to other
suggestions, such as providing more statistical data or uses
specific, useful libraries. One interesting comment was made
in regards to our possibility of adding additional GUI to the
framework. They said that they did not think this was
necessary of helpful, which was useful information for us as
developers. Another suggestion was to be sure to include
samples or example of how to run the framework, including
details such specifying the key length.

%%%%%%%%%%%%%%%%%%%%%%%%%%%%%%%%%%%%%%%%%%%%%%%%%%%%%%%%%%%%
