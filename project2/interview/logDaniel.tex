\subsection{Interviewer: \dd}

%%%%%%%%%%%%%%%%%%%%%%%%%%%%%%%%%%%%%%%%%%%%%%%%%%%%%%%%%%%%

\begin{answer}{1}
Yes
\end{answer}

%%%%%%%%%%%%%%%%%%%%%%%%%%%%%%%%%%%%%%%%%%%%%%%%%%%%%%%%%%%%

\begin{answer}{2}
Yes
\end{answer}

%%%%%%%%%%%%%%%%%%%%%%%%%%%%%%%%%%%%%%%%%%%%%%%%%%%%%%%%%%%%

\begin{answer}{3}
Yes, it would be helpful.
   The report should include a statement about whether the cryptographic strength being evaluated is weak, adequate, or strong. This could be accompanied by a numeric value. For example, this algorithm is a 6 on a scale of 1-10.
   Another feature that would be helpful would be a statement of whether the algorithm could be broken by something other than brute force. Was there a glaring weakness in the encryption method that made it so that a brute force attack is unnecessary?
   A last feature might be a statement about what effect key size has on the algorithm being evaluated. That is, would a slightly larger key size greatly increase the strength of the encryption?
\end{answer}

%%%%%%%%%%%%%%%%%%%%%%%%%%%%%%%%%%%%%%%%%%%%%%%%%%%%%%%%%%%%

\begin{answer}{4}
AES encryption support is desired. SHA-2 (and possibly SHA-1) hash function support is desired. In addition, legacy encryption and hash function support would be helpful. Many legacy encryption algorithms and hash functions are still in use. Legacy encryption algorithms desired are RC4, RC5, and 3DES. Legacy hash functions desired are MD4, MD5, and SHA-1. RC4 and MD4 are specified since these may still be in use on old Windows systems like Windows 2000. Also, many people are still in the process of upgrading from SHA-1 to SHA-2.
\end{answer}

%%%%%%%%%%%%%%%%%%%%%%%%%%%%%%%%%%%%%%%%%%%%%%%%%%%%%%%%%%%%

\begin{answer}{5}
Assuming that Cry finds a problem with the encryption function, the following feature would be desirable. This feature would be a determination of whether the algorithm can be quickly strengthened (e.g. just increase key size), whether a moderate amount of work is needed to needed to strengthen the encryption (e.g. quickly swap out encryption algorithm), or a determination that a complete rewrite is needed.
\end{answer}

%%%%%%%%%%%%%%%%%%%%%%%%%%%%%%%%%%%%%%%%%%%%%%%%%%%%%%%%%%%%

\begin{answer}{6}
The best way to improve the framework is by having one or more examples which illustrate how to use that framework. Even though the interview question did not show the command line prompts for input, make sure the question is clear with an example of the answer. The following might be sample prompt,
“Size of key (enter a number, for example 50)”.
\end{answer}

%%%%%%%%%%%%%%%%%%%%%%%%%%%%%%%%%%%%%%%%%%%%%%%%%%%%%%%%%%%%
