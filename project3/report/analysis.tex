\section{System Analysis and Decomposition}

%%%%%%%%%%%%%%%%%%%%%%%%%%%%%%%%%%%%%%%%%%%%%%%%%%%%%%%%%%%%

Cry will consist of two main objects: Cryptosystem and
Party. Party will interact with Cryptosystem to generate
keys, encrypt plaintext, decrypt ciphertext, and run
cryptanalysis on ciphertext.
\medskip

Party will consist of three subclasses: Receiver, Sender,
and Eavesdropper. Receiver will generate keys and decrypt
messages sent by Sender. Receiver's data members will be:
a public publicKey, a private privateKey, and a private
plaintext. Sender will encrypt messages to be sent to
Receiver. Sender's data members will be: a public
ciphertext, and a private plaintext. Eavesdropper will
attempt to derive the key from Receiver's publicKey and
Sender's ciphertext, while also having access to the name
of the cryptosystem being used by all parties.
\medskip

Cryptosystem will be an abstract class that we use as the
template for all cryptosystems.
\medskip

We will implement a concrete class of Cryptosystem called
RsaCryptosystem. RsaCryptosystem will be an implementation
of the familiar RSA encryption scheme. It's cryptanalyze
function will make use of General Number Field Sieve's to
make cracking more efficient than mere brute force.
\medskip

Cryptosystem will be an enemurated type used by Party to
identify the cryptosystem.
\medskip

Text will be a user-defined type that is simply a string.
\medskip

Key will be a user-defined type that will contain an
mpz\_t type. mpz\_t is defined in the GNU Multiple Precision
Arithmetic Library. mpz\_t allows integers of unrestricted
size, up to the size of addressable memory.
