\section{GMP with \cpp}

\subsection{Overview}

\begin{frame}
\frametitle{my frame title}
\begin{itemize}
\item GMP is natively written in \c
\item most \c{} constructs can be used in \cpp
\item GMP \cpp{} interface is convenient:
  \begin{itemize}
  \item templates
  \item classes
  \item overloaded functions/operators
  \end{itemize}
\end{itemize}
\end{frame}

\subsection{Example}

\begin{frame}
\frametitle{my frame title}
File: \code{tutorial.h}
\codes{\lstinputlisting[language=c++]{../code/vu/tutorial.h}}
\end{frame}

\begin{frame}
\frametitle{my frame title}
File: \code{tutorial.cc}
\codes{\lstinputlisting[language=c++]{../code/vu/tutorial.cc}}
\end{frame}

\begin{frame}
\frametitle{my frame title}
File: \code{main.cc}
\codes{\lstinputlisting[language=c++]{../code/vu/main.cc}}
\end{frame}

\begin{frame}
\frametitle{my frame title}
my frame body
\end{frame}

%%%%%%%%%%%%%%%%%%%%%%%%%%%%%%%%%%%%%%%%%%%%%%%%%%%%%%%%%%%%

\section{GMP with \make}

\subsection{Overview}

\begin{frame}
\frametitle{my frame title}
\begin{itemize}
\item GNU \make{} is a tool that determines
  which parts of a large project need to be recompiled
\item user writes a \code{makefile} to specify
  the dependencies among source files
\end{itemize}
\end{frame}

\subsection{Example}

\begin{frame}
\frametitle{my frame title}
File: \code{makefile}
\codes{\lstinputlisting[language=make]{../code/vu/makefile}}
\end{frame}
