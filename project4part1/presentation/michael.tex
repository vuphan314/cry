\section{GMP Functions}

%%%%%%%%%%%%%%%%%%%%%%%%%%%%%%%%%%%%%%%%%%%%%%%%%%%%%%%%%%%%

\subsection{Integer Initialization}

\begin{frame}
\frametitle{Two basic initialization techniques}
  \begin{itemize}
    \item Initialize Only\break
    mpz\_init(mpz\_t);
    \item Initialize and Assign\break
    mpz\_init\_set\_ui(mpz\_t, unsigned long int);
  \end{itemize}
\end{frame}

%%%%%%%%%%%%%%%%%%%%%%%%%%%%%%%%%%%%%%%%%%%%%%%%%%%%%%%%%%%%

\subsection{Integer Arithmetic}

\begin{frame}
\frametitle{Integer Addition/Subtraction}
  \begin{itemize}
    \item Add two GMP integers\break
    mpz\_add(mpz\_t, const mpz\_t, const mpz\_t);
    \item Add a GMP integer and a literal\break
    mpz\_add\_ui(mpz\_t, const mpz\_t, unsigned long int);
    \item Multiply two GMP integers\break
    mpz\_mul(mpz\_t, const mpz\_t, const mpz\_t);
    \item Multiply a number by power of 2\break
    mpz\_mul\_2exp(mpz\_t, const mpz\_t, mp\_bitcnt\_t);
    \item Take the modulo of two GMP integers\break
    mpz\_mod(mpz\_t, const mpz\_t, const mpz\_t);
    \item Compare a GMP integer and a literal\break
    mpz\_cmp\_ui(const mpz\_t, unsigned long int);
    \item Get the length of a GMP number in a cerain base\break
    size\_t mpz\_sizeinbase(const mpz\_t, int);
  \end{itemize}
\end{frame}

%%%%%%%%%%%%%%%%%%%%%%%%%%%%%%%%%%%%%%%%%%%%%%%%%%%%%%%%%%%%

\subsection{Other Functions}

\begin{frame}
\frametitle{Random Numbers}
  \begin{itemize}
    \item Set the random number algorithm to be used\break
    gmp\_randinit\_default(gmp\_randstate\_t);
    \item Set the random number seed to a literal\break
    gmp\_randseed\_ui(gmp\_randstate\_t, unsigned long int);
    \item Get a random integer up to a certain bit-length\break
    mpz\_urandomb(mpz\_t, gmp\_randstate, mp\_bitcnt\_t);
  \end{itemize}
\end{frame}

\begin{frame}
\frametitle{Prime Numbers/Other}
  \begin{itemize}
    \item Get the next prime number larger than a GMP integer\break
    mpz\_nextprime(mpz\_t, const mpz\_t);
    \item Take the inverse modulo of two GMP integers\break
    int mpz\_invert(mpz\_t, const mpz\_t, const mpz\_t);
  \end{itemize}
\end{frame}

\begin{frame}
\frametitle{What do you mean "inverse modulo"?}
  \begin{itemize}
    \item 0+5=5\break
    5+(-5)=0\break
    0 is the additive identity and -5 is the additive inverse of 5
    \item 1*5=5\break
    5*(1/5)=1\break
    1 is the multiplicative identity and (1/5) is the multiplicative inverse of 5
    \item 3*3 $\equiv$ 1 (mod 4)\break
    3 is a modular inverse of 3 taken in the context of (mod 4)\break
    So is 7 and 11\break
    In fact, the formula to calculate a modular inverse of 3 taken
    in the context of (mod 4) is X=(3+4n) where n is an integer
  \end{itemize}
\end{frame}
