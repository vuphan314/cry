\section{GMP Functions}

%%%%%%%%%%%%%%%%%%%%%%%%%%%%%%%%%%%%%%%%%%%%%%%%%%%%%%%%%%%%

\subsection{Integer Initialization}

\begin{frame}
\frametitle{Two basic initialization techniques}
\begin{itemize}
  \item Initialize Only
  mpz\{\_}init(mpz\{\_}t);
  \item Initialize and Assign
  mpz\{\_}init\{\_}set\{\_}ui(mpz\{\_}t, unsigned long int);
\end{itemize}
\end{frame}

%%%%%%%%%%%%%%%%%%%%%%%%%%%%%%%%%%%%%%%%%%%%%%%%%%%%%%%%%%%%

\subsection{Integer Arithmetic}

\begin{frame}
\frametitle{Integer Addition/Subtraction}
\begin{itemize}
  \item Add two GMP integers
  mpz{\_}add(mpz{\_}t, const mpz{\_}t, const mpz{\_}t);
  \item Add a GMP integer and a literal
  mpz{\_}add{\_}ui(mpz{\_}t, const mpz{\_}t, unsigned long int);
  \item Multiply a number by power of 2
  mpz{\_}mul{\_}2exp(mpz{\_}t, const mpz{\_}t, mp{\_}bitcnt{\_}t);
  \item Take the modulo of two GMP integers
  mpz{\_}mod(mpz{\_}t, const mpz{\_}t, const mpz{\_}t);
  \item Compare a GMP integer and a literal
  mpz{\_}cmp{\_}ui(const mpz{\_}t, unsigned long int);
  \item Get the length of a GMP number in a cerain base
  size{\_}t mpz{\_}sizeinbase(const mpz{\_}t, int);
\end{itemize}
\end{frame}

%%%%%%%%%%%%%%%%%%%%%%%%%%%%%%%%%%%%%%%%%%%%%%%%%%%%%%%%%%%%

\subsection{Other Functions}

\begin{frame}
\frametitle{Random Numbers}
\begin{itemize}
  \item Set the random number algorithm to be used
  gmp{\_}randinit{\_}default(gmp{\_}randstate{\_}t);
  \item Set the random number seed to a literal
  gmp{\_}randseed{\_}ui(gmp{\_}randstate{\_}t, unsigned long int);
  \item Get a random integer up to a certain bit-length
  mpz{\_}urandomb(mpz{\_}t, gmp{\_}randstate, mp{\_}bitcnt{\_}t);
\end{frame}

\begin{frame}
\frametitle{Prime Numbers/Other}
\begin{itemize}
  \item Get the next prime number larger than a GMP integer
  mpz{\_}nextprime(mpz{\_}t, const mpz{\_}t);
  \item Take the inverse modulo of two GMP integers
  int mpz{\_}invert(mpz{\_}t, const mpz{\_}t, const mpz{\_}t);
\end{itemize}
\end{frame}

\begin{frame}
\frametitle{What do you mean "inverse modulo"?}
\begin{itemize}
  \item 0+5=5 5+(-5)=0
  0 is the additive identity and -5 is the additive inverse of 5
  \item 1*5=5 5*(1/5)=1
  1 is the multiplicative identity and (1/5) is the multiplicative inverse of 5
  \item 3*3 \equiv 1 (mod 4)
  3 is a modular inverse of 3 taken in the context of (mod 4)
  So is 21
  In fact, the formula to calculate a modular inverse of 3 taken
  in the context of (mod 4) is X=(9+12n) where n is an integer
\end{itemize}
\end{frame}
