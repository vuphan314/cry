\section{GMP Functions}

%%%%%%%%%%%%%%%%%%%%%%%%%%%%%%%%%%%%%%%%%%%%%%%%%%%%%%%%%%%%

% \subsection{Integer Initialization}
%
% \begin{frame}
% \frametitle{Two basic initialization techniques}
% \begin{itemize}
%   \item Initialize Only
%   mpz_init(mpz_t);
%   \item Initialize and Assign
%   mpz_init_set_ui(mpz_t, unsigned long int);
% \end{itemize}
% \end{frame}

%%%%%%%%%%%%%%%%%%%%%%%%%%%%%%%%%%%%%%%%%%%%%%%%%%%%%%%%%%%%

\subsection{Integer Arithmetic}

\begin{frame}
\frametitle{Integer Addition/Subtraction}
\begin{itemize}
  \item Add two GMP integers
  mpz_add(mpz_t, const mpz_t, const mpz_t);
  \item Add a GMP integer and a literal
  mpz_add_ui(mpz_t, const mpz_t, unsigned long int);
  \item Multiply a number by power of 2
  mpz_mul_2exp(mpz_t, const mpz_t, mp_bitcnt_t);
  \item Take the modulo of two GMP integers
  mpz_mod(mpz_t, const mpz_t, const mpz_t);
  \item Compare a GMP integer and a literal
  mpz_cmp_ui(const mpz_t, unsigned long int);
  \item Get the length of a GMP number in a cerain base
  size_t mpz_sizeinbase(const mpz_t, int);
\end{itemize}
\end{frame}

%%%%%%%%%%%%%%%%%%%%%%%%%%%%%%%%%%%%%%%%%%%%%%%%%%%%%%%%%%%%

\subsection{Other Functions}

\begin{frame}
\frametitle{Random Numbers}
\begin{itemize}
  \item Set the random number algorithm to be used
  gmp_randinit_default(gmp_randstate_t);
  \item Set the random number seed to a literal
  gmp_randseed_ui(gmp_randstate_t, unsigned long int);
  \item Get a random integer up to a certain bit-length
  mpz_urandomb(mpz_t, gmp_randstate, mp_bitcnt_t);
\end{frame}

\begin{frame}
\frametitle{Prime Numbers/Other}
\begin{itemize}
  \item Get the next prime number larger than a GMP integer
  mpz_nextprime(mpz_t, const mpz_t);
  \item Take the inverse modulo of two GMP integers
  int mpz_invert(mpz_t, const mpz_t, const mpz_t);
\end{itemize}
\end{frame}

\begin{frame}
\frametitle{What do you mean "inverse modulo"?}
\begin{itemize}
  \item 0+5=5 5+(-5)=0
  0 is the additivie identity and -5 is the additive inverse of 5
  \item 1*5=5 5*(1/5)=1
  1 is the multiplicative identity and (1/5) is the multiplicative inverse of 5
  \item 3*3 \equiv 1 (mod 4)
  3 is a modular inverse of 3 taken in the context of (mod 4)
  So is 21
  In fact, the formula to calculate a modular inverse of 3 taken
  in the context of (mod 4) is X=(9+12n) where n is an integer
\end{itemize}
\end{frame}
