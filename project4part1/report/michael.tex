\section{GMP Functions}

% If you want to show code from another file:
% \codes{\lstinputlisting[language=c++]{../code/michael/michael.cc}}
% path relative to `project4part1/report/`

\url{https://gmplib.org/manual/Integer-Functions.html#Integer-Functions}

%%%%%%%%%%%%%%%%%%%%%%%%%%%%%%%%%%%%%%%%%%%%%%%%%%%%%%%%%%%%

\subsection{Integer Initialization}

\url{https://gmplib.org/manual/Initializing-Integers.html#Initializing-Integers}
\url{https://gmplib.org/manual/Simultaneous-Integer-Init-_0026-Assign.html#Simultaneous-Integer-Init-_0026-Assign}

mpz\_init(mpz\_t);\break
This function will simply initialize an mpz\_t variable.
mpz\_t variables cannot be used if they are not initialized
first.
\medskip
mpz\_init\_set\_ui(mpz\_t, unsigned long int);\break
This function simultaneously initializes an mpz\_t variable
and assigns it the value of a regular unsigned long integer.

%%%%%%%%%%%%%%%%%%%%%%%%%%%%%%%%%%%%%%%%%%%%%%%%%%%%%%%%%%%%

\subsection{Integer Arithmetic}

mpz\_add(mpz\_t, const mpz\_t, const mpz\_t);\break
The second and third arguments are summed and stored in the
first argument.
\medskip
mpz\_add\_ui(mpz\_t, const mpz\_t, unsigned long int);\break
This is the same as the above function, except the third
argument is a regular unsigned long integer.
\medskip
mpz\_mul(mpz\_t, const mpz\_t, const mpz\_t);\break
The second and third arguments are multiplied together and
stored in the first argument.
\medskip
mpz\_mul\_2exp(mpz\_t, const mpz\_t, mp\_bitcnt\_t);\break
The second argument is multiplied by two to the power of
the third argument and stored in the first argument.
\medskip
mpz\_mod(mpz\_t, const mpz\_t, const mpz\_t);
The value of the second argument modulo the third argument
is stored in the first argument.

\subsection{Integer Comparisons}

\url{https://gmplib.org/manual/Integer-Comparisons.html#Integer-Comparisons}

int mpz\_cmp\_ui(const mpz\_t, unsigned long int);\break
If the first argument is greater than the second, returns
a positive value. If the first argument is less than the
second, returns a negative value. If the first argument is
equal to the second, returns 0.

\subsection{Random Numbers}

\url{https://gmplib.org/manual/Random-State-Initialization.html#Random-State-Initialization}
\url{https://gmplib.org/manual/Integer-Random-Numbers.html#Integer-Random-Numbers}

gmp\_randinit\_default(gmp\_randstate\_t);\break
Initializes a gmp\_randstate\_t variable to the default
random number generator provided by GMP (compromise between
speed and randomness).
\medskip
gmp\_randseed\_ui(gmp\_randstate\_t, unsigned long int);\break
Sets the seed of the gmp\_randstate\_t variable to a regular
unsigned long integer.
\medskip
mpz\_urandomb(mpz\_t, gmp\_randstate, mp\_bitcnt\_t);\break
Generates a random number using the algorithm denoted by the
second argument, where the third argument specifies the
maximum number of bits the number should contain and stores
it in the first argument.

\subsection{Prime Numbers}

\url{https://gmplib.org/manual/Number-Theoretic-Functions.html#Number-Theoretic-Functions}

mpz\_nextprime(mpz\_t, const mpz\_t);\break
This function will find the first prime number greater than
the second argument and store the value in the first
argument.

\subsection{Special Functions}

\url{https://gmplib.org/manual/Integer-Special-Functions.html#Integer-Special-Functions}

size\_t mpz\_sizeinbase(const mpz\_t, int);\break
This function returns the number of characters needed to
represent the first argument, given the base of the second
argument.

\subsection{Theoretic Functions}

\url{https://gmplib.org/manual/Number-Theoretic-Functions.html#Number-Theoretic-Functions}

int mpz\_invert(mpz\_t, const mpz\_t, const mpz\_t);\break
The value of the second argument inverse modulo the third
argument is stored in the first argument.
