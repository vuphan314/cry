\section{\cry{} Description}

If you want to show code from another file:
\codes{\lstinputlisting[language=c++]{../code/daniel/daniel.cc}}
% path relative to `project4part1/report/`


The goal of our project is to develop a framework for
cryptographers to easily test their \cs s in a
secure environment. Users will be able to implement their
\cs{} securely in Cry, and upon testing it, they will
receive a report that will explain results of a particular
test as well as ways to improve the system.

\medskip

This report will be a useful tool for cryptographers. It
will give details on various aspects of the tested system.
These details will include time (or estimated time) required
to encrypt, decrypt, and cryptanalyze. When analyzing a
cryptanalysis method, Cry will hope to give some useful
suggestions as to why the system was broken in the given
time, and what could be used to improve it in further
iterations.

%%%%%%%%%%%%%%%%%%%%%%%%%%%%%%%%%%%%%%%%%%%%%%%%%%%%%%%%%%%%

\section{GMP Installation}

\url{https://gmplib.org/manual/Installing-GMP.html#Installing-GMP}

Initially, a download is obviously needed. For Linux systems, one can simply go to the url
given above, and download the .tar file. This does not work for Mac. Instead, following the
instructions we have posted on our Github, GMP installation can be done using the mathit{brew install gmp}
command.

On Linux systems, after downloading the file, a simple mathit{./configure} and then mathit{make install}
will install GMP.

Some things to keep in mind when builing. One can build to a separate directory that they want to use or
are already currently using for their project. In addition, one option that is sometimes needed is to either
disable shared or disable static. Sharing libraries can sometimes help with the performance if the CPU is
capable, but it can slow down if it is not capable. It is suggested to configure GMP for the specific CPU that
you are running on your machine, as this will give the best performance. The trade off is that it will have worse performance
if it is migrated to another machine, even using a newer CPU.

Other customizations for GMP can be used by building for different packages, instead of the original package, depending on your needs.
Another call to performance is to build the included file call mathit{tune} as a preliminary step if the intended program is demanding extremely large numbers or plans to run for long periods of time.

%%%%%%%%%%%%%%%%%%%%%%%%%%%%%%%%%%%%%%%%%%%%%%%%%%%%%%%%%%%%

\section{GMP Basics}

If the 2 sections above are too short,
then you may want to add this section.

\url{https://gmplib.org/manual/GMP-Basics.html#GMP-Basics}
