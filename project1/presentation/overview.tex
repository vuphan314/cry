\section{Overview}
\frame{\sectionpage}

%%%%%%%%%%%%%%%%%%%%%%%%%%%%%%%%%%%%%%%%%%%%%%%%%%%%%%%%%%%%

\subsection{Example Use Case}

\frame{\subsectionpage}

\begin{frame}
\frametitle{Situation}
\begin{itemize}
\item
  \alice{} (sender) wants to confidentially send a message
  to \bob{} (receiver).
\item
  \eve{} (eavesdropper) wants to know that message.
\end{itemize}
\end{frame}

\begin{frame}
\frametitle{Procedure}
\begin{itemize}
\item
  \bob{} generates his keys: \getkeysex
\item
  \alice{} (sender) \encryptex
\end{itemize}
\end{frame}

\begin{frame}
\frametitle{Procedure}
\begin{itemize}
\item
  \bob{} (receiver) \decryptex
\item
  \eve{} (eavesdropper) \eavesdropex
\end{itemize}
\end{frame}

%%%%%%%%%%%%%%%%%%%%%%%%%%%%%%%%%%%%%%%%%%%%%%%%%%%%%%%%%%%%

\subsection{Functionality}

\frame{\subsectionpage}

\begin{frame}
\frametitle{In the previous example:}
\begin{itemize}
\item
  \cry{} is the \cf.
\item
  RSA is a \cs{} implemented in \cry.
\item
  The key-generation, encryption, decryption,
  and eavesdropping algorithms are specific to RSA.
\end{itemize}
\end{frame}

\begin{frame}
\frametitle{In general, with \cry:}
\begin{itemize}
\item an \eu{} can use an implemented \cs{}
  to confidentially send and receive messages with others.
\item a \cg{} can:
  \begin{itemize}
  \item prototype her own \cs s
    where the \ca s are either newly defined
    or reused from different existing \cs s.
  \item test her \cs s for security and performance.
  \end{itemize}
\end{itemize}
\end{frame}
