% \eu shorthand for end-users
% \cry shorthand for our system
% \cs shorthand for cryptosystem
% \cg shorthand for cryptographers

\section{Motivation}

\begin{itemize}
\item For \eu s:
  \begin{itemize}
  \item Problem: An \eu{} wants to send and receive
    secure messages with other \eu s.
  \item Solution: \cry{} lets \eu s establish
    secure communication via built-in \cs s.
  \end{itemize}
\item For \cg s:
  \begin{itemize}
  \item Problem: As quantum computing becomes more prevalent,
  new \cs s need to be developed that are traditionally secure,
  as well as secure in the quantum world.
    \begin{itemize}
    \item Reason: Existing \cs s are secure
      only if their assumptions are true.
    \item Example: The RSA \cs's security relies on
      the assumption that integer factorization is hard.
      It is currenly unknown whether a polynomial-time
      factorizing algorithm exists for traditional binary
      computers. A polynomial-time factorizing algorithm
      does, however, exist for quantum computers.
    \item Example: The DES \cs's security relied on the assumption
      that it was not practical to attack by brute force it's key space
      (approximately 72 quadrillion keys). In 1998, the EFF (Electronic
      Frontier Foundation) built Deep Crack at a cost of approximately
      250,000 USD (a very affordable fee), which successfully brute
      forced DES in 56 hours.
    \item Example: The AES \cs's security also relies on the assumption
      that it is not practical to attack by brute force it's key space.
      This assumption should prove to be correct for the forseable future.
      AES when implemented correctly is not susceptible to brute force attacks,
      nor is it susceptible to any known cryptanalysis techniques.
    \end{itemize}
  \item Solution: \cry{} let \cg s easily prototype, test,
    and benchmark their new \cs s.
  \end{itemize}
\end{itemize}
