\section{Overview}

%%%%%%%%%%%%%%%%%%%%%%%%%%%%%%%%%%%%%%%%%%%%%%%%%%%%%%%%%%%%

\subsection{Example Use Case}

Situation:
\begin{itemize}
\item
  \alice{} (sender) wants to confidentially send a message
  to \bob{} (receiver).
\item
  \eve{} (eavesdropper) wants to know that message.
\end{itemize}

Procedure:
\begin{enumerate}
\item
  Each person downloads the binary file \code{cry}
  of the \cry{} \cf.
\item
  \bob{} publishes his choice of \cs:
  RSA (Rivest, Shamir, Adleman).
\item
  \bob{} generates his keys: \getkeysex
\item
  \bob{} publishes his public key
  (and hides his private key).
\item
  \alice{} obtains \bob's published public key.
\item
  \alice{} encrypts her message (say, her phone number):
  \encryptex
\item
  \alice{} publishes the encrypted message.
\item
  \bob{} obtains \alice's published encrypted message.
\item
  \bob{} easily decrypts the message with his private key:
  \decryptex
\item
  \eve{} struggles to eavesdrop the message
  without \bob's private key: \eavesdropex
\end{enumerate}

%%%%%%%%%%%%%%%%%%%%%%%%%%%%%%%%%%%%%%%%%%%%%%%%%%%%%%%%%%%%

\subsection{Functionality}

In the previous example:
\begin{itemize}
\item
  \cry{} is the \cf.
\item
  RSA is a \cs{} implemented in \cry.
\item
  The key-generation, encryption, decryption,
  and eavesdropping algorithms are specific to RSA.
\end{itemize}

In general, with \cry:
\begin{itemize}
\item an \eu{} can use an implemented \cs{}
  to confidentially send and receive messages with others.
\item a \cg{} can:
  \begin{itemize}
  \item prototype her own \cs s
    where the \ca s are either newly defined
    or reused from different existing \cs s.
  \item test her \cs s for security and performance.
  \end{itemize}
\end{itemize}
