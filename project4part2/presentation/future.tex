\section{Remaining Work}

%%%%%%%%%%%%%%%%%%%%%%%%%%%%%%%%%%%%%%%%%%%%%%%%%%%%%%%%%%%%

\begin{frame}
\frametitle{}
\begin{itemize}
\item \cry{} users are \eu s and \cg s.
\item For \eu s, we implemented the \rsa{} \cs.
\item For \cg s, we will implement the \dummy{} \cs:
  \begin{itemize}
  \item \dummy{} will serve as a simplistic example
    on how to create a new \cs{} using our \cry{} \cf.
  \item \dummy{} is for demonstration purposes only,
    so \dummy{} needs not be secure.
  \item We will try to make \dummy{} an asymmetric \cs{}
    (like \rsa) instead of a symmetric \cs{} (like \aes).
  \end{itemize}
\item We are on track to complete \cry{} within the allotted
  time budget. If we are unable to develop a suitable
  asymmetric \cs{}, we will develop a symmetric \cs{}
  instead.
\end{itemize}
\end{frame}

\begin{frame}
\frametitle{}
\begin{itemize}
  \item To develop \dummy{}, we need to follow a basic
    process.
  \item First, we need to identify a suitable invertible
    function, such that the function's inverse is
    non-trivial.
  \item Second, we need to devise an algorithm to generate
    a private key, and a public key.
  \item As a bad example, we could select the multiplication
    function. We could choose the public key 5, and the
    private key 1/5. We multiply the plaintext by 5 to get
    the ciphertext, and we multiply the ciphertext by 1/5 to
    get the plaintext.
\end{itemize}
\end{frame}
