\section{Finished Work}

\begin{frame}
\frametitle{}
The \cs{} \rsa{} has been implemented in the \cf{} \cry.
\end{frame}

%%%%%%%%%%%%%%%%%%%%%%%%%%%%%%%%%%%%%%%%%%%%%%%%%%%%%%%%%%%%

\subsection{Terminology}

\begin{frame}
\frametitle{}
Each \textdef{key element} is a positive integer:
\begin{itemize}
\item $n$: \textdef{modulus}
\item $e$: \textdef{public exponent}
\item $d$: \textdef{private exponent}
\end{itemize}
\end{frame}

\begin{frame}
\frametitle{}
Each \textdef{key} is a pair of key elements:
\begin{itemize}
\item $(n, e)$: \textdef{public key}
\item $(n, d)$: \textdef{private key}
\end{itemize}
\end{frame}

\begin{frame}
\frametitle{}
Each \textdef{padded text} is a nonnegative integer:
\begin{itemize}
\item $m$: \textdef{padded plaintext}
\item $c$: \textdef{padded ciphertext}
\end{itemize}
\end{frame}

\begin{frame}
\frametitle{}
Each (unpadded) \textdef{text} is a string. \\
Our padding simply means we convert a string to an integer
\\
For more information, refer to our project 4 progress report
at: \\ \url
{https://github.com/vuphan314/cry/releases/tag/v0.4.0-alpha}
\end{frame}

%%%%%%%%%%%%%%%%%%%%%%%%%%%%%%%%%%%%%%%%%%%%%%%%%%%%%%%%%%%%

\subsection{Key Generation}

\begin{frame}
\frametitle{Michael}
\begin{itemize}
\item Input: none
\item Output: $((n, e), (n, d))$ (public key, private key)
\item Algorithm:
  \begin{enumerate}
  \item choose some distinct prime numbers $p$ and $q$
  \item let $n = p * q$
  \item let $l$ be the least common multiple of
    $p - 1$ and $q - 1$
  \item choose some $e$ such that:
  \begin{itemize}
    \item $1 < e < l$
    \item $e$ and $l$ are coprime
    \item In our case, we set $e$ to the decimal integer
      65537. This guarantees that $e$ and $l$ are coprime,
      makes the modular exponentiation quick, and is very
      common practice.
  \end{itemize}
  \item let $d$ be the modular inverse of $e$
    modulo $l$
  \item return $((n, e), (n, d))$
  \end{enumerate}
\end{itemize}
\end{frame}

%%%%%%%%%%%%%%%%%%%%%%%%%%%%%%%%%%%%%%%%%%%%%%%%%%%%%%%%%%%%

\subsection{Encryption}

\begin{frame}
\frametitle{Daniel}
\begin{itemize}
\item Input: $(m, (n, e))$ (padded plaintext, public key)
\item Output: $c$ (padded ciphertext)
\item Algorithm:
  \begin{enumerate}
  \item let $c = m^e$ modulo $n$
  \item return $c$
  \end{enumerate}
\end{itemize}
\end{frame}

%%%%%%%%%%%%%%%%%%%%%%%%%%%%%%%%%%%%%%%%%%%%%%%%%%%%%%%%%%%%

\subsection{Decryption}

\begin{frame}
\frametitle{Daniel}
\begin{itemize}
\item Input: $(c, (n, d))$ (padded ciphertext, private key)
\item Output: $m$ (padded plaintext)
\item Algorithm:
  \begin{enumerate}
  \item let $m = c^d$ modulo $n$
  \item return $m$
  \end{enumerate}
\end{itemize}
\end{frame}

%%%%%%%%%%%%%%%%%%%%%%%%%%%%%%%%%%%%%%%%%%%%%%%%%%%%%%%%%%%%

\subsection{Cryptanalysis}

\begin{frame}
\frametitle{Vu}
\begin{itemize}
\item Input: $(c, (n, e))$ (padded ciphertext, public key)
\item Output: $m$ (padded plaintext)
\item Algorithm:
  \begin{enumerate}
  \item find $d$:
    \begin{enumerate}
    \item factor $n = p * q$ where
      $p$ and $q$ are prime numbers \\
      ($n$ was created this way in the key-generation step)
    \item let $l$ be the least common multiple of
      $p - 1$ and $q - 1$
    \item let $d$ be the multiplicative inverse of $e$
      modulo $l$
    \end{enumerate}
  \item compute $m$:
    \begin{enumerate}
    \item let $m = c^d$ modulo $n$
    \item return $c$
    \end{enumerate}
  \end{enumerate}
\end{itemize}
\end{frame}

%%%%%%%%%%%%%%%%%%%%%%%%%%%%%%%%%%%%%%%%%%%%%%%%%%%%%%%%%%%%

\subsection{Demo}

\begin{frame}
\frametitle{}
\lstinputlisting[basicstyle=\small, lastline=9]{../log.txt}
\end{frame}

\begin{frame}
\frametitle{}
\lstinputlisting[basicstyle=\small, firstline=10]{../log.txt}
\end{frame}
