\section{Finished Work}

The \cs{} \rsa{} has been implemented in the \cf{} \cry.

\cite{RSA}

%%%%%%%%%%%%%%%%%%%%%%%%%%%%%%%%%%%%%%%%%%%%%%%%%%%%%%%%%%%%

\subsection{Terminology}

Each \textdef{key element} is a positive integer:
\begin{itemize}
\item $n$: \textdef{modulus}
\item $e$: \textdef{public exponent}
\item $d$: \textdef{private exponent}
\end{itemize}

Each \textdef{key} is a pair of key elements:
\begin{itemize}
\item $(n, e)$: \textdef{public key}
\item $(n, d)$: \textdef{private key}
\end{itemize}

Each \textdef{padded text} is a nonnegative integer:
\begin{itemize}
\item $m$: \textdef{padded plaintext}
\item $c$: \textdef{padded ciphertext}
\end{itemize}

Each (unpadded) \textdef{text} is a string.
Padding and unpadding were already described in
the project 4 progress report at: \\ \url
{https://github.com/vuphan314/cry/releases/tag/v0.4.0-alpha}

%%%%%%%%%%%%%%%%%%%%%%%%%%%%%%%%%%%%%%%%%%%%%%%%%%%%%%%%%%%%

\subsection{Key Generation}

% Michael:
% How did I implement this function?

\begin{itemize}
\item Input: none
\item Output: $((n, e), (n, d))$
\item Algorithm:
  \begin{enumerate}
  \item choose some distinct prime numbers $p$ and $q$
  \item let $n = p * q$
  \item let $l$ be the least common multiple of
    $p - 1$ and $q - 1$
  \item choose some $e$ such that:
  \begin{itemize}
    \item $1 < e < l$
    \item $e$ and $l$ are coprime
  \end{itemize}
  \item let $d$ be the multiplicative inverse of $e$
    modulo $l$
  \item return $((n, e), (n, d))$
  \end{enumerate}
\end{itemize}

%%%%%%%%%%%%%%%%%%%%%%%%%%%%%%%%%%%%%%%%%%%%%%%%%%%%%%%%%%%%

\subsection{Encryption}

% Daniel:
% How did I implement this function?

\begin{itemize}
\item Input: $(m, (n, e))$
\item Output: $c$
\item Algorithm:
  \begin{enumerate}
  \item let $c = m^e$ modulo $n$
  \item return $c$
  \end{enumerate}
\end{itemize}

%%%%%%%%%%%%%%%%%%%%%%%%%%%%%%%%%%%%%%%%%%%%%%%%%%%%%%%%%%%%

\subsection{Decryption}

% Daniel:
% How did I implement this function?

\begin{itemize}
\item Input: $(c, (n, d))$
\item Output: $m$
\item Algorithm:
  \begin{enumerate}
  \item let $m = c^d$ modulo $n$
  \item return $m$
  \end{enumerate}
\end{itemize}

%%%%%%%%%%%%%%%%%%%%%%%%%%%%%%%%%%%%%%%%%%%%%%%%%%%%%%%%%%%%

\subsection{Cryptanalysis}

% Vu:
% How did I implement this function?

\begin{itemize}
\item Input: $(c, (n, e))$
\item Output: $m$
\item Algorithm:
  \begin{enumerate}
  \item find $d$:
    \begin{enumerate}
    \item factor $n = p * q$ where
      $p$ and $q$ are prime numbers \\
      ($n$ was created this way in the key-generation step)
    \item let $l$ be the least common multiple of
      $p - 1$ and $q - 1$
    \item let $d$ be the multiplicative inverse of $e$
      modulo $l$
    \end{enumerate}
  \item compute $m$:
    \begin{enumerate}
    \item let $m = c^d$ modulo $n$
    \item return $m$
    \end{enumerate}
  \end{enumerate}
\end{itemize}

%%%%%%%%%%%%%%%%%%%%%%%%%%%%%%%%%%%%%%%%%%%%%%%%%%%%%%%%%%%%

\subsection{Demo}

\codes{\lstinputlisting{../log.txt}}
