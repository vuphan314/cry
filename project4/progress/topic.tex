\section{Technical Topic}
\dd \bigskip

% 2. What is the topic for your technical presentation?
% Explain briefly.

The GMP library contains several topics useful for the
situations and calculations we will need in our project.

\begin{itemize}

\item Integer Functions: many basic functions for integer
  arithmetic such as division, exponentiation, roots, as
  well as other miscellaneous functions. Note, all integers
  are stored as type mpz\_t.

\item Rational Number Functions: an important function of
  rational numbers is canonicalizing them, which is to
  remove any common factors in the numerator and
  denominator, i.e. reduce them. Note, rational numbers are
  stored as type mpq\_t.

\item Floating-point Functions: floating points arithmetic
  in GMP is a main feature of the library. Mantissas of each
  number has user-selected precision, and so is limited only
  by computer memory. Similar to regular IEEE double, the
  mantissa is stored in binary   which makes decimal
  fractions, such as 0.1, difficult. The exponent precision
  is fixed, and is normally the size of a word in standard
  machines. As such, values of computations may differ on
  some machines if the size of that machine's word is
  different. In addition, there are not special
  notations to indicate infinity or other ``non-numbers''.
  Note, floating point numbers are stored as type mpf\_t.

\item Low-level Functions: functions designed for optimal
  speed and performance but without the coherent interface
  of the higher-level functions. Note, functions here are
  preluded with mpn\_.

\item Random Number Functions: normal sequence of
  pseudo-random generated numbers which can be integers or
  floating point. Note, all types stored as
  gmp\_randstate\_t.

\item Formatted Output.
\item Formatted Input.
\item C++ Class Interface.
\item Custom Allocations.
\item Language Bindings.
\item Algorithms.

\end{itemize}

%%%%%%%%%%%%%%%%%%%%%%%%%%%%%%%%%%%%%%%%%%%%%%%%%%%%%%%%%%%%
